\documentclass[12pt]{article}
\usepackage[utf8]{inputenc}
\usepackage[russian]{babel}
\usepackage{amsmath}
\usepackage{amssymb}
\usepackage{hyperref} 

\numberwithin{equation}{section} 
\begin{document}
    \tableofcontents 
    \newpage
    \section{Средняя vs. максимальная сложность}
      Определим насколько сильно могут различаться среднее время вычисления конкретной булевой функции и ее сложность, т.~е. время вычисления в худшем случае. Положим
$$m(f) = L(f)/T(f), \quad m(n) = \max m(f),$$
где максимум берется по всем функциям, зависящим от
$n$~переменных.
Прежде всего отметим, что в~силу теорем <<Метод Лупанова>>, <<Монотонные функции>> и <<Средняя сложность почти всех функций>> при $n~\xrightarrow~\infty$ для почти каждой $n$~---~местной булевой функции $f$~справедливы неравенства
$$ 2\lesssim m(f) \lesssim 8,$$
т.~е. для почти каждой булевой функции досрочное прекращение вычиcлений позволяет уменьшить объем вычислений не более чем в~конечное число раз. С~другой стороны, пример симметрических пороговых функций показывает, что такое уменьшение может быть более значительным. Максимально возможное значение такого уменьшения оценивается в следующей теореме, где при $n~\xrightarrow~\infty $ устанавливается порядок роста функции $m(n)$.
\subsection{Первая теорема} \newtheorem{Th}{Теорема} \begin{Th}
Существуют такие постоянные $c_1$ и $c_2$, что
$$c_1{\left(\frac{2^n}{n}\right)}^{1/2}\le m(n) \le c_2{\left(\frac{2^n}{n}\right)}^{1/2}.$$ \end{Th}

{\sc Доказательство.} Пусть 
$ k = \lceil (n+\log n)/2 \rceil $ 
и $g$~---~самая сложная булева функция от $k$~переменных, т.~е.
$L(g) = \Theta {(2^n/n)}^{1/2}$. Рассмотрим функцию
$$ f(x_1,\ldots,x_n) = \bar x_{k+1}\& \cdots \& \bar x_n\&g(x_1,\ldots,x_n). $$
Пусть минимальная программа $P(g)$ вычисляет функцию $g$. Тогда программа $P$, начинающаяся с команд
$$\begin{tabular}{l l}
      $p_1$ : & z=$0$ \\
      $p_2$ : & {Stop}$(x_{k+1})$\\
      $p_3$ : & {Stop}$(x_{k+2})$\\
      \ldots \ldots & \ldots \ldots \ldots \ldots \\
      $p_{n-k+1}$ : & {Stop$(x_n)$}, \\
\end{tabular}$$
после которых идут команды программы $P(g)$, вычисляет $f$. Легко видеть, что
$$T(P)\le 2^{-n} \left(\sum_{j=1}^{n-k}(j+1)2^{n-j}+(n-k+1+C(g))2^k \right) = O(1),$$
и $L(f) = \Theta(2^n/n)^{1/2}$. Следовательно, $m(n) \ge c_1(2^n/n)^{1/2}$.

Теперь покажем, что
$m(n) \le c_2(2^n/n){1/2}$. Пусть
$f$~---~произвольная булева функция от $n$ переменных, $P$~---~программа, которая вычисляет
$f$, и среднее время ее работы минимально.

Положим $ k = \lfloor (n+\log n)/2 \rfloor $. Рассмотрим набор
$x$ такой, что $N_P(x) =
2^n - 2^k$. Так как
$$T(P) = 2^{-n}\sum_y T_P(y)>2^{-n} \quad \sum_{y | N(y)>N(x)}T_P(y)\ge2^{-n}2^k T_P(x),$$
то легко видеть, что
\begin{equation}
    2^{k-n}T_P(x)<T(f).\label{eq.simple}
\end{equation}
Далее, пусть $\tilde f$~---~частичная булева функция, определенная на всех таких наборах
$y_i$, что $N_P(y_i) > N_P(x)$, и совпадающая на этих наборах с
$f$. Так как $2^k = \Theta(2^n n)^{1/2}$, то из теоремы <<Частичные функции>> следует существование схемы $S$,
вычисляющей $\tilde f$, и такой, что
\begin{equation}L(S) = O \left(\frac{2^n}{n} \right)^{1/2}.
\label{eq.simple 1}\end{equation}
Теперь опишем программу $P'$, вычисляющую функцию $f$. Сначала воспользуемся программой $P$, которая за минимальное среднее время вычисляет $f$. С ее помощью будем вычислять значения функции $f$ на наборах $y$
таких, что $N_P(y) \le
N_P(x)$. Так как $2^{n-k} =
O\left(2^n/n\right)^{1/2}$, то из~\eqref{eq.simple} следует, что для вычисления функции $f$ на этих наборах потребуется выполнить не
более $H_1$ команд программы $P$, где
\begin{equation}H_1 = O\left( \left(\frac{2^n}{n} \right)^{1/2}T(f) \right).
\label{eq.simple 2}\end{equation}
Для вычисления функции $f$ на оставшихся наборах воспользуемся неветвящейся программой--схемой, реализующей функцию $\tilde f$. Очевидно, что эта программа $S$ содержит не более $H_2$ команд, где
\begin{equation}H_2 = O\left( \left(\frac{2^n}{n} \right)^{1/2} \right),
\label{eq.simple 3}\end{equation}
что следует из \eqref{eq.simple 1}.

Таким образом, из \eqref{eq.simple 2}
и \eqref{eq.simple 3} следует, что сложность
$C(P')$ программы $P'$ по порядку не превосходит величины
$$\left(\frac{2^n}{n} \right)^{1/2}T(f) + \left(\frac{2^n}{n} \right)^{1/2} \le 2\left(\frac{2^n}{n} \right)^{1/2}T(f). $$
Так как $L(f) =
O(C(P'))$, то найдется константа $c_2$ такая, что
$$L(f)/T(f)\le c_2 \left(\frac{2^n}{n} \right)^{1/2}.$$
Теорема доказана.

Теперь покажем, что для каждой булевой функции $f$ существует программа, сложность и среднее время работы которой одновременно близки, соответственно, к~сложности в~худшем случае и к средней сложности этой функции.
\subsection{Вторая теорема}
\newtheorem{The}{Теорема}
\begin{Th} Для любой булевой функции
$f$ найдется такая вычисляющая ее программа $P$, что
$$T(P)\le 2T(f), \quad C(P)\le 4C(f).$$ 
\end{Th}

{\sc Доказательство.} Пусть
$P$~---~программа, вычисляющая функцию $f$ за
минимальное среднее время. Пусть $x$~---~набор с минимальным номером такой, что $T_P(x) \ge 2C(f)$ (если такого набора нет, то утверждение теоремы
тривиально, так как тогда $T(f)=T(P)\le C(P) \le 2C(f)$). Ясно, также что $T_P(x) \le 3C(f)$, поскольку невыполнение этого неравенства влечет существование в программе $P$ расположенных друг за другом $C(f)+1$ вычислительных команд, что противоречит минимальности программы $P$. Среднее
время работы программы $P$ можно представить следующим образом:
\begin{equation*}
\begin{split}
 T(P)&= 2^{-n} \left( \sum_{N_P(y)<N_P(x)}T_P(y)+ \sum_{N_P(y)\ge N_P(x)}T_P(y) \right)=\\ & = T_1+T_2\ge T_1+2C(f)(2^n-N_P(x)+1)2^{-n} .
\end{split}
\end{equation*}
Преобразуем программу $P$, заменив команды с номерами большими $T_P(x)$ программой без команд остановки~---~минимальной схемой из функциональных элементов вычисляющей $f$. Легко видеть, для сложности новой программы $P'$ справедливо неравенство
$$ C(P')\le T_P(x)+C(f)\le 4C(f), $$
а для среднего времени работы этой программы~---~неравенство:
\begin{equation*}{\begin{split}
 T(P')&\le 2^{-n} \left( \sum_{N_P(y)<N_P(x)}T_{P'}(y)+ \sum_{N_P(y)\ge N_P(x)}T_{P'}(y) \right)=\\ & = T'_1+T'_2\le T'_1+4C(f)(2^n-N_P(x)+1)2^{-n} \le \\ & \le
 2(T'_1+2C(f)(2^n-N_P(x)+1)2^{-n}).
\end{split}}
\end{equation*}
Так как $T'_1=T_1$, то $T(P')\le 2T(P)$. Теорема доказана.

\end{document}